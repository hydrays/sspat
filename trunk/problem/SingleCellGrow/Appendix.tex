%\documentclass[12pt, draft]{article}
\documentclass[12pt]{article}
\usepackage{amscd}
\usepackage{amssymb}
\usepackage{mathrsfs}
\usepackage{amsfonts}
\usepackage{mathtools}

\pagestyle{plain}
%\addtolength{\voffset}{1cm}

\usepackage[active]{srcltx}
\usepackage{amsmath,amssymb,amsthm,amsfonts}
%\usepackage{times}
\usepackage{enumerate}
\usepackage{epsfig}
\usepackage{graphicx}
\usepackage{pgf}
\usepackage{subfigure}
%\usepackage{setspace}

\newcommand{\bfmath}[1]{{\mathchoice
    {\mbox{\boldmath$\displaystyle#1$}}
    {\mbox{\boldmath$\textstyle#1$}}
    {\mbox{\boldmath$\scriptstyle#1$}}
    {\mbox{\boldmath$\scriptscriptstyle#1$}}}}
\usepackage[T1]{fontenc}
\usepackage[latin1]{inputenc}
\usepackage[a4paper]{geometry}
%\usepackage{lmodern}
\usepackage[english]{babel}

\newcommand\T{\rule{0pt}{2.6ex}}
\newcommand\B{\rule[-1.2ex]{0pt}{0pt}}

\newcommand{\E}{\mathbb{E}}

\newtheorem{algo}{Algorithm}
\newtheorem*{algo*}{Algorithm}
\newtheorem{remark}{Remark}

\newcommand{\order}{\mathcal{O}}
\textwidth=6.0in
\oddsidemargin=.25in

\def\thesection{\Roman{section}} 

\begin{document}

\title{Supplemental Information}

\author{Yucheng Hu, John Lowengrub, Arthur Lander}
\date{}
\maketitle

\subsection*{A stochastic cell lineage model}
Consider a cell population consists of SCs, TACs and TDCs. 
At time $t$ the system state is denoted by 
$\bfmath{X}(t) = \left( x_0, x_1, x_2 \right)$,
where $x_0, x_1, x_2$ represent the number of SCs, TACs and TDCs, respectively.
SC and TAC may divide, while TDC lives for a certain period and die. 
Denote the division rate of SC and TAC to be $v_0$ and $v_1$ 
and the death rate of TDC to be $a$.
A SC may divide into two SCs with probability $q_0$, 
or one SC and one TAC with probability $r_0$, 
or two TACs with probability $1-q_0-r_0$.
Similarly for TACs, we have $q_1$, $r_1$, $1-q_1-r_1$ as the probability 
of one TAC divides into two TACs, one TAC and one TDC, 
and two TDCs, respectively. 

So if $n$ SCs divide, there will be $2nq_0+nr_0$ SCs 
and $nr_0+2n(1-q_0-r_0)$ TACs on average. 
So on the population level, the self-renewal probability 
of SC as the ratio of SCs in the cells produced by SCs 
from the previous generation, or 
\begin{equation}
p_0 = \frac{2nq_0 + nr_0}{2n} = q_0+\frac{r_0}{2}.
\label{eq_popp0}
\end{equation} 
So the self-renewal probability of TACs is
\begin{equation}
p_1 = \frac{2nq_1 + nr_1}{2n} = q_1 + \frac{r_1}{2}.
\label{eq_popp1}
\end{equation}
$p_0$ and $p_1$ are the effective self-renewal probability 
of SC and TAC, respectively.

To model the limit capacity of the environment that the cells live in, 
we assume the total population is controlled by a "soft ceiling'' $N$.
We call $N$ as the carrying capacity, and 
only when the total population $n=x_0+x_1+x_2 > N$, 
the environment will induce a overall death rate $d$
to all the cells. We choose 
\begin{equation}
a = \max \left( 0, \xi \left( \frac{n}{N} - 1\right) \right).
\label{eq_d}
\end{equation} 
So the real population $n$ can be larger 
than the carrying capacity $N$, but when this happens,
the environmental induced overall death rate will 
pull the total population back towards $N$.
Here the parameter $\xi>0$ controls the 
rigidness of this pulling force.
This underlying population model is similar 
with the ``ceiling model'' that has been used 
in ecology~\cite{Lindenmayer95}.

The above population evolutionary process 
can be described as a chemical reaction system.
\begin{eqnarray*}
\mathrm{SC} &\xrightarrow[]{\makebox[2.0cm]{\scriptsize{$v_0 q_0$}}}& 
\mathrm{SC} + \mathrm{SC} \\
\mathrm{SC} &\xrightarrow[]{\makebox[2.0cm]{\scriptsize{$v_0 r_0$}}}& 
\mathrm{SC} + \mathrm{TAC} \\
\mathrm{SC} &\xrightarrow[]{\makebox[2.0cm]{\scriptsize{$v_0 (1-q_0-r_0)$}}}& 
\mathrm{TAC} + \mathrm{TAC} \\
\mathrm{TAC} &\xrightarrow[]{\makebox[2.0cm]{\scriptsize{$v_1 q_1$}}}& 
\mathrm{TAC} + \mathrm{TAC} \\
\mathrm{TAC} &\xrightarrow[]{\makebox[2.0cm]{\scriptsize{$v_1 r_1$}}}& 
\mathrm{TAC} + \mathrm{TDC} \\
\mathrm{TAC} &\xrightarrow[]{\makebox[2.0cm]{\scriptsize{$v_1 (1-q_1-r_1)$}}}& 
\mathrm{TDC} + \mathrm{TDC} \\
\mathrm{TDC} &\xrightarrow[]{\makebox[2.0cm]{\scriptsize{$d$}}}& \emptyset \\
\mathrm{SC} &\xrightarrow[]{\makebox[2.0cm]{\scriptsize{$a$}}}& \emptyset \\
\mathrm{TAC} &\xrightarrow[]{\makebox[2.0cm]{\scriptsize{$a$}}}& \emptyset \\
\mathrm{TDC} &\xrightarrow[]{\makebox[2.0cm]{\scriptsize{$a$}}}& \emptyset
\end{eqnarray*}


The reactions involving the mutated cell lineage are:
\begin{eqnarray*}
\mathrm{MC} &\xrightarrow[]{\makebox[2.0cm]{\scriptsize{$v_m q_m$}}}& 
\mathrm{MC} + \mathrm{MC} \\
\mathrm{MC} &\xrightarrow[]{\makebox[2.0cm]{\scriptsize{$v_m r_m$}}}& 
\mathrm{MC} + \mathrm{MTDC} \\
\mathrm{MC} &\xrightarrow[]{\makebox[2.0cm]{\scriptsize{$v_m (1-q_m-r_m)$}}}& 
\mathrm{MTDC} + \mathrm{MTDC} \\
\mathrm{MTDC} &\xrightarrow[]{\makebox[2.0cm]{\scriptsize{$d$}}}& \emptyset \\
\mathrm{MC} &\xrightarrow[]{\makebox[2.0cm]{\scriptsize{$a$}}}& \emptyset \\
\mathrm{MTDC} &\xrightarrow[]{\makebox[2.0cm]{\scriptsize{$a$}}}& \emptyset \\
\end{eqnarray*}

\subsection*{Feedback regulation}
We assume feedback regulation exists in our model 
which comes from the TDC population and regulate 
the behavior of SC by negatively controlling its 
self-renewal probability $p_0$ and division rate $v_0$. 
Mathematically, this means $p_0$ and $v_0$ are functions
of the TDC population $x_2$. In particular, for $p_0$ we choose
\begin{equation}
p_0 = \frac{1}{1.01+k_0 x_2 / N},
\label{feedp0}
\end{equation}
where $k_0$ controls the strength of the feedback. 
The constant 1.01 in the denominate is to make $p_0<1$ even when $x_2=0$, 
which allow SCs still have a slight chance to differentiate 
even when there is no TDC.

For $v_0$ we consider two cases: in one case we fix $v_0$ as constant,
which means the division rate does not response to feedback regulation;
in the other we let $v_0$ vary between $v_0^{min}$ and $v_0^{max}$
as $x_2$ varying from $1$ to $0$. This can be done by choosing
the feedback strength $l_0$ in the following equation
\begin{equation*}
v_0 = \frac{v_0^{max}}{1+l_0 x_2/N},
\end{equation*}
as
$$
l_0 = v_0^{max}\left( \frac{1}{v_0^{min}} - \frac{1}{v_0^{max}}\right),
$$
which gives
\begin{equation}
v_0 = \left( \frac{1}{v_0^{max}} + \frac{x_2}{N}
\left( \frac{1}{v_0^{min}} - \frac{1}{v_0^{max}}\right) \right)^{-1},
\label{feedv0}
\end{equation}

It should be noted that $p_0$ that is being regulated
is the effective self-renewal probability. After it being
fixed, we still have freedom in choosing 
$q_0$ (1SC -> 2SC) and $r_0$ (1SC -> 1SC + 1TAC)
as long as $p_0 = q_0 + r_0/2$.
This is similar for TAC in choosing $q_1$ and $r_1$ that
satisfies $q_1 + r_1/2 = p_1$.
This gives us certain amount of freedom in 
choosing between symmetrical division 
and asymmetrical division. In the deterministic case, 
this makes no difference, 
but in the stochastic case, especially when the number of SCs is small, 
symmetrical division cause more fluctuation, 
and SC population is more likely to extinguish stochastically 
(discussed in the appendix).
In the following, we will assume maximum possible asymmetrical division. 
So for a given value of $p_0$, if $p_0 \le 0.5$, there is no need for 
symmetrically division (because asymmetrical division already 
has self-renewal probability equals 0.5), so we can choose 
$q_0=0$ and $r_0 = 2p_0$.
For $p_0>0.5$, only asymmetrical division can 
not generate enough SCs, so we must at least have 
$q_0 = 2p_0 - 1$ and $r_0 = 2(1-p_0)$. 
For TAC it is the same.

\subsection*{Feedback on $p_0$ regulates the population structure}
First we consider the simplest case where there is only
one feedback on the self-renewal probability of SC $p_0$ 
as~\eqref{feedp0} and fix $v_0=1$. $p_1=0.4$ and $v_1=1$ 
are also fixed. Other parameters are listed in Table~\ref{table_paras}. 

\begin{table}
\centering
\caption{Parameters in the stochastic cell lineage model.}
\begin{tabular}{|c|c|c|}
\hline
Symbol & Value & Description \\
\hline
$N$ & 200 & carrying capacity \\
\hline
$\xi$ & 20 & controls the induced-death rate when $n>N$ \\  
\hline
$v_1$ & 1 & division rate of TAC \\
\hline
$p_1$ & 0.4 & self-renewal probability of TAC \\
\hline
$d$ & 0.2 & apoptosis rate of TDC \\
\hline
$v_0^{min}$ & 0.5 & lower bound of SC division rate \\
\hline
$v_0^{max}$ & 3 & upper bound of SC division rate \\
\hline
\end{tabular}
\label{table_paras}
\end{table}

Initially we put 10 SCs in the system ($\mathbf{X}_0 = (10, 0, 0)$) 
and let the system evolve under the stochastic cell lineage model. 
Since there is no TDC to provide feedback, 
the SC population expands fast with self-renewal probability close to 1. 
Soon the total population hit the carrying capacity $N$ 
and the induced death rate $d$ will prevent the population 
from continue growing. Meanwhile, since $p_0$ is less then 1, 
some SCs will differentiate into TAC, 
which will continue differentiate into TDC. 
The TDCs negatively regulate $p_0$ making more
SCs to differentiate. 
Eventually the population fluctuates around a 
equilibrium state, in which the SC proliferate just
enough to account for a slightly loose due to the
environmental induced death.

Figure~\ref{fig_pop_struct} shows some typical simulation trajectories
of the system with only one feedback on $p_0$. 
In (a) when $k_0$ is small, the number of SCs is large
and the number of TDCs is small. 
In (b) when $k_0$ is large, the number of SCs becomes
small and the number of TDCs becomes large. (c) and (d) 
show the value of $p_0$ ($v_0=1$ is constant in this case).
(e) and (f) shows the average number of each species for
different value of $k_1$. It is the time average of population
from one simulation start at time $t=200$ and end at time $t=500$.

A system may benefit from having a 
population structure with a small number of
SC supporting a medium population of TAC, which support 
a large number of TDC. One possible benefit
is that the division burden has shifted from
SC to TAC so that number of SC can stay low,
hence reduce the risk of mutation.

Then we consider another feedback on $v_0$ together
with the feedback on $p_0$.
It has been shown that some SCs will remain in 
a slow-dividing quiescent state in homogeneous state.
When the TDCs suffer a great lose, 
for example in injury, they can transit 
to a fast-dividing activated state under. 
To account for the change in division rate of SC,
we add another feedback regulation to $v_0$ as in~\eqref{feedv0}.
Under this feedback, as the number of TDC $x_2$ changes between 0 and 1, 
$v_0$ will change between $v_0^{max}$ and $v_0^{min}$.
$v_0^{max}$ and $v_0^{min}$ are considered as parameters 
of the model and their values are shown in Table~\ref{table_paras}.

With two feedback on $p_0$ and $v_0$
the system can also maintain a stable population structure.
Figure~\ref{fig_pop_struct_two} shows that, 
similar with the one feedback case,
starting from the initial condition $\mathbf{X}_0 = (10, 0, 0)$,
the system reaches a equilibrium state after a transit increase of
SC. However, (a) and (c) show that when $k_1$ the feedback strength
on $p_0$ is small, the number of SCs is large and the division 
rate of SC remains very high, meaning the SCs are in activated state. 
While in (b) and (d), the number of SCs remains to be low as well as
$v_0$ ($v_0$ being around 0.65). Mimicking the quiescent state of SC.
Also, from (f) we can see that as $k_1$ grows, similar
with Fig.~\ref{fig_pop_struct}, the average number of 
SCs increases, but there is a sharp transit region between
the activated state and quiescent state.

\subsection*{Feedback on $v_0$ improves genetic stability}
If we assume that the mutation rate for each cell is 
proportional to its division rate, then by staying at
a quiescent state the SC gain another protection from
mutations. Interestingly, numerical results 
show that the activation of SC into a more competitive
fast-dividing state provides an extra defense mechanism 
to some deleterious mutations.

We consider a mutation type which is similar with TAC.
This mutant cell has fixed self-renewal probability $p_m=1$ and 
division rate $v_m=1$, which are constants and
are not regulated by feedback. So the mutant 
cell divides as fast as the TAC without differentiation but only proliferate
itself. This kind of mutant cell can be considered to have
a deleterious phenotype. The mutant
type will proliferate and out-competing with the normal cell lineage. 
In the system with only one feedback on $p_0$, even thought the SC
can sense the reduction of TDC (via feedback) and responses by increase
self-renewal probability $p_0$, still it can not match the mutant cell
whose $p_m=1$. But in the system with two feedback on $p_0$ and $v_0$,
as the number of TDC decreases, the SC can switch to an active state
with $p_0$ close to 1 and $v_0$ close to $v_0^{max} > v_m$. Now the SC
is more fit than the mutant type and it may kill the mutant specie entirely.
After the mutant type is clear from the system, SC begin to differentiate
and can return to the quiescent state due to the feedback regulation.
The process that the system self-recovery from deleterious mutations
is depicted in Figure~\ref{fig_cartoon} (the left part). 

Figure~\ref{fig_self_recover} is a numerical simulation showing the 
self-recovery process. It shows one typical trajectory of the time 
evolution of the system regulated by feedback on $p_0$ and $v_0$.
Starting with the same initial condition $(10, 0, 0)$,
  the system reaches a equilibrium state, in which the number of SC
is kept low and both the self-renewal probability $p_0$ and 
division rate $v_0$ are small. At time $t=200$, we manually put
one mutant cell with phenotype $p_m=1$ and $v_m=1$ in to the system.
The mutant type soon proliferate and as its number grows bigger other
cell types are shrinking in population size. The decrease of the number
of TDC drives both $p_0$ and $v_0$ up, thus activating the quiescent 
SC population. The activated SC has $p_0$ close to 1 and $v_0>v_m$, thus
to be more fit than the mutant cell. As a result, it not only stops the 
mutant population from growing, but also completely drive the entire 
mutant population to extinguish. Later on, without the mutant cell anymore,
the SC will differentiate and the system can recovery to its previous
equilibrium state under the feedback regulations.
 
Next we ask how the system will response to
other mutant cell type with different $p_m$ and $v_m$.
Assume when a mutant cell dividing, 
with probability $q_m$ and $r_m$ it generates two mutant cells and
one mutant cell one differentiated mutant cell, respectively, and
with probability $1-q_m-r_m$ it generates two differentiated mutant cells.
We also favor asymmetrical division over symmetrical division, such that
if $p_m \le 0.5$, $q_m=0$ and $r_m = 2p_m$ and if 
$p_m>0.5$, $q_m = 2p_m - 1$ and $r_m = 2(1-p_m)$.
Plus, they also share the same induced death rate $d$ with all the others.

Each fixed pair of parameters $(p_m, v_m)$ 
determines one mutant type. At time $t=200$, when 
the system in a equilibrium state, we put one such mutant
cell into the system, and then simulate the system to as long
as $t=2000$. By then the system may be mutant cell free or 
has been taken over by mutant cells, or coexist by mutant cells 
and normal lineage. For each mutant type, we repeat the simulation
1000 times, and record which of the three possible outcome occurs
and the maximum number of SC population ever reached in each run.

Using the above numerical equilibrium, we studied the response to
mutant type in two systems: the one feedback only on $p_0$
and the two feedback on $p_0$ and $v_0$. We choose 
fixed division rate $v_0=0.65$ in the first system so that the two 
systems have the same equilibrium state. 
Other parameters in the two systems are 
still the same as Table~\ref{table_paras}. The results 
are shown in Figure~\ref{fig_fixpmvm}.

For the first system, the mutation takeover and SC response are shown
in Figure~\ref{fig_fixpmvm} (a)-(f). 
Each point in (a) corresponding to a fixed
parameters pair $(p_m, v_0)$. The possible value range of $p_m$ is $[0, 1]$,
but since for $p_m<0.5$, the mutant type cannot self-sustain, so only the
range $[0.5,1.0]$ is shown. The range of $v_m$ is set as 
$[v_0^{min}=0.5, v_0^{max}=3.0$. The darkness in the figure corresponds
to the number of takeover instance out from the 1000 simulated samples.
In (b) it shows the average of the maximum number of SC reached during 
each simulation. The colored value is the fold on increasing of the number
of SC compared with the number of SC in equilibrium state. 
This value indicates how the system response to the 
mutant cell. The higher this value is, the more red is the color,
and more intense is the system response. For this system, the region
of $(p_m, v_m)$ that the mutant cell will takeover is fairly large.
The system can make limited response when the mutant cell is not 
too deleterious. In fact, the degree of deleterious of a mutant type
corresponds to the product $v_m (p_m - 0.5)$. For each point 
in (b), we compute this product and use this value as the x-axis
in (c), and the fold on increase of the SC number as the y-axis.  
It can been seen that the response of the system is a function of
the product $v_m (p_m - 0.5)$. (d)-(f) are typical trajectories
obtained by choosing $p_m, v_m$ at points A, B, C, respectively.

For the second system, the mutation takeover and SC response
are shown in Figure~\ref{fig_fixpmvm} (g)-(l). 
For this system, the region
of $(p_m, v_m)$ that the mutant cell will takeover is greatly reduced.
Meanwhile, the system can make much more aggressive 
response for mutant cells as a function of
the product $v_m (p_m - 0.5)$ ((b) and (c)).
It shows the system with two feedbacks are more robust than
the one with one feedback.

\subsection*{Differentiation as a carcinogenesis strategy}
In the second system studied above, 
the mutant cell has a probability $1-p_m$
to differentiate into some kind of terminal cell, 
but the differentiated mutant cell does not feedback on 
SC like the normal TDC does. 
Now we consider the possibility that the differentiated mutant cell
can feedback on SC as the same way a normal TDC does.
It turns out this gives the mutant type a great advantage
in taking over the system.

Let $x_3$ be the number of mutant cells and $x_4$ be the 
number of differentiated mutant cells in the system,
the feedback relations on $p_0$ and $v_0$ now should be,
\begin{equation}
p_0 = \frac{1}{1.01+k_0 (x_2+x_4) / N},
\label{feedp0e}
\end{equation}
\begin{equation}
v_0 = \left( \frac{1}{v_0^{max}} + \frac{x_2+x_4}{N}
\left( \frac{1}{v_0^{min}} - \frac{1}{v_0^{max}}\right) \right)^{-1},
\label{feedv0e}
\end{equation}

The numerical results are shown in Figure~\ref{fig_fixpmvm} (m)-(r).
In (m) there is a great increase in size of the region that
takeover is possible, especially for small $p_m$ and $v_m$.
Another major difference is as $p_m$ get close to 1, the
takeover region is actually shrinking. In (n) we can see 
that this is because these mutant cell trigger quite large
SC response and the normal lineage out-compete the mutant type. 
In fact, the mutant cell with large $p_m$   
are too busy proliferating and they do not generate 
enough terminal cells to negatively feedback on SCs, 
which are activated and stop mutant cell from taking over 
the system. Indeed, with moderate division rate, 
the best strategy for mutant cell to survive is not proliferate 
by brute force. Instead, a certain level of differentiation 
to produce a differentiated cell is more effective. 
This result may also suggest that most of the cancer 
cells are heterogeneous in population structure rather 
than a homogeneous composition of some super cancer cell.

\subsection*{How mutant cell escapes from the feedback regulation}
So far we have been assuming fixed phenotype $p_m$ and $v_m$ for
the mutant cell. One can consider these cells are mutants from TACs,
which do not response to feedback regulation just as the TACs do.
If a mutation occur in the SC population, it is likely that
the mutant cell will experience some change in its sensitivity
in responding feedback, rather than lose feedback completely.
We ask what kind of change in feedback sensitivity help
the mutant cell to breakdown the normal lineage system.

In a system with feedback on both $p_0$ and $v_0$, 
we assume mutant cell has $p_m$ and $v_m$ determined
by the following feedback functions similar 
with those of SC's given by~\eqref{feedp0e} and~\eqref{feedv0e}.
$$
p_m = \frac{1}{1.01+k_m (x_2 + x_4)/N}, \ \ \ 
v_m = \left( \frac{1}{v_m^{max}} + \frac{x_2+x_4}{N}
\left( \frac{1}{v_m^{min}} - \frac{1}{v_m^{max}}\right) \right)^{-1}, 
$$
We fixed the parameter $v_m^{min}=v_0^{min}0.5$ and change 
$k_m$ and $v_m^{max}$. Thus for each $(k_m, v_m^{max})$ 
pair, the feedback function $p_m$ and $v_m$ is determined
and so does the phenotype of the mutant cell. For 
this mutant cell, we do the same numerical experiment as above to
see how the system response to it. That is, at $t=200$
we manually introduce one mutant cell to the system and 
simulate the system state to $t=2000$ to see if: (i) 
the mutant type has taken over the system; or (ii) 
the mutant type has extinguished; or (iii) the mutant 
type and cells from the original lineage system coexistent.

A sample size of 1000 simulations are made for each parameter-pair
$(k_m, v_m^{max})$ ranging from $[0, 1.5]\times[0.5, 4]$.
Note that the SC has $k_0 = 1$, $v_0^{min}=v_m^{min}=0.5$,
and $v_0^{max}=3$, so this gives us four regions with the 
properties that $\mathrm{I} \rightarrow (p_m \le p_0, v_m \le v_0)$, 
$\mathrm{II} \rightarrow (p_m \le p_0, v_m > v_0)$, 
$\mathrm{III} \rightarrow (p_m > p_0, v_m \le v_0)$, 
$\mathrm{IV} \rightarrow (p_m > p_0, v_m > v_0)$, 
respectively. Figure~\ref{fig_feedpmvm} (a) shows,
for the case that the differentiated mutant cell does not 
feedback, how the system response to different mutant cells.
Compared with the normal SC, 
the phenotype of mutant cell in in region I
is less fit in both the self-renewal channel and 
the division rate channel.
It shows that these mutant cells cannot survive.
In region II and region IV, the mutant cell has 
one advantageous channel and one disadvantageous channel.
Only a small portion of mutant cell may takeover when 
the advantage gain is strong while the 
disadvantage lose is not too weak.
In region III, both channels are advantageous, making 
the mutant cell more fit than SC so all mutant cell in 
this region will takeover. The SC response in (b) shows
how feedback regulation can actively protect the system
by activate the SC population. Things are quite different
if the differentiated mutant cell can feedback as a normal
TDC. By doing so, it allows the mutant cell to takeover 
much easier in region III. By comparing with the SC response
in (d), we can see this is done by suppress the activation of
SC through feedback. Even more interesting, 
the takeover region shows a finger like shape.
In fact, when the $k_m$ is around 0.8, even when a
very disadvantage in the $v_m$ channel presents, the 
mutant cell can still takeover. 

%% \subsection*{Experimental Support}
%% In breast cancer, there are subtypes called luminal A, 
%% luminal B, HER2b/ER, basal-like, and normal breast-like. 
%% They have distinctive biochemical, morphological and 
%% growth properties. The basal-like subtype has been 
%% shown to have the highest proliferation rates and poor 
%% clinical outcomes~\cite{Livasy06} and the basal-like 
%% subtype is more likely to be found in the 
%% poorly-differentiated grade III~\cite{Benporath08}.
%% Similar relationship is found in brain tumor, 
%% that the Glioblastoma (88h), Astrocytoma III (180h) 
%% and Astrocytoma II (544h) have decreased growth rate 
%% (cell cycle time is shown in the brackets)~\cite{Schroder91}.
%% In general, it has been proposed that increased cell 
%% division as a cause of human cancer~\cite{Prestonmartin90}.

%% In figure 11~\cite{Fillmore08}, a number of breast cancer 
%% cell lines is cultured in vitro. We can see the cell 
%% growth rates are related to their stem-like cell ratio. 
%% For example, the cell lines with higher proliferation 
%% rate, SUM159 and SUM149, indeed contain more stem-like cells, 
%% and cell lines with lower proliferation rate, MCF7 and SUM225, 
%% indeed contain less stem-like cells. 
%% Also, there is a strong connection between growth rate 
%% and CFE (clonal forming efficiency). 
%% Similar behavior is also observed in brain tumor~\cite{Galli04}. 
%% In all, these data matches well with our prediction. 
%% We make couple of remarks here.

%% \begin{remark}
%% When cultured in certain conditions in vitro, stem-like 
%% and nonstem-like cell populations has similar growth rate. 
%% This means the overall growth rate indeed reflect the 
%% growth rate of the CSC sub-population. 
%% Figure 12 from~\cite{Gupta11} provides a support. 
%% In general there might be heterogeneous in the division 
%% rate among stem-like and nonstem-like cells. 
%% In this case we need to actually estimate the division 
%% rate of CSC instead of the overall growth rate. 
%% Precise experimental data with this kind of information 
%% is still missing. Still, there are other evidence from 
%% the cell morphology study and genetic expression 
%% study~\cite{Benporath08, Ross00} that 
%% the poorly-differentiated cancer cells tend to divide 
%% faster than the well-differentiated cancer cells. 
%% \end{remark}

%% \begin{remark}
%% Under ideal culture condition, we believe the CFE is correlated 
%% with (a) the ratio of colony-forming cells (or stem-like cells) 
%% in the seeding population, and (b) the self-renewal probability 
%% of the stem-like cells. Since both (a) and (b) are positively 
%% correlated with the self-renewal probability of the CSC $p_m$, 
%% we claim CFE is also positively correlated with $p_m$. 
%% So according to our prediction, the faster the CSC in a 
%% cell line divides, the higher CFE. 
%% This is consistent with Figure 11 for breast cancer cell lines. 
%% More evidence can be found in, 
%% such as, Esophageal cancer~\cite{Shimada91}.
%% \end{remark}

\bibliography{ref}
\bibliographystyle{plain}

\section*{Figure Captions}
\begin{itemize}
\item 
Figure 1. Cell lineage and feedback. We model the health cell lineage
as SC (stem cell) -> TAC (transit amplifying cell) -> TDC (terminal
differentiated cell). SC and TAC can divide while TDC will live for
an exponentially distributed time then die. The cell divisions are 
stochastic with determined by the probability given in the figure.
We study the feedback control from TDC to SC, which controls the 
self-renewal probability and division rate of SC. Later we will 
also consider the mutation cells. Mutation can happen either in SC
or TAC. The mutation cell can also self-renewal and differentiation, 
thus have its own lineage.

\item 
Figure 2. Feedback strength on $p_0$ regulates 
the population structure.

\item
Figure 3. Self-recovery from deleterious mutations. 

\item
Figure 4. Response to mutant cells with different phenotype (illustration).

\item
Figure 5. Response to mutant cells with different phenotype 
(numerical simulation).

\item
Figure 6. Feedback in mutant cells.
\end{itemize}

\section*{Appendix}

\subsection*{The deterministic model}
Denote the number of cells of SC, TAC, TDC, MC and MTDC as
$x_0, x_1, x_2, x_3, x_4$, respectively. 
The ODE for the cell lineage model in Fig.~\ref{fig_feedback} 
with mutation cell is
\begin{eqnarray*}
\dot{x}_0 &=& q_0 v_0 x_0 - (1-q_0-r_0)v_0x_0 - a x_0, \nonumber \\
\dot{x}_1 &=& r_0 v_0 x_0 + 2 (1-q_0-r_0) v_0 x_0 + 
q_1 v_1 x_1 - (1-q_1-r_1) v_1 x_1 - a x_1, \nonumber \\
\dot{x}_2 &=& r_1 v_1 x_1 + 2 (1-q_1-r_1) v_1 x_1 - d x_2 - a x_2, \nonumber \\
\dot{x}_3 &=& q_m v_m x_3 - (1-q_m-r_m) v_m x_3 - a x_3, \nonumber \\
\dot{x}_4 &=& r_m v_m x_3 + 2 (1-q_m-r_m) v_m x_3 - d x_4 - a x_3.
%\label{ode}
\end{eqnarray*}
With the relation in Eq.~\eqref{eq_popp0} and Eq.~\eqref{eq_popp1},
and rescale $x_i$ by the carrying capacity $z_i = x_i/N, i=0,1,2,3,4$, 
the above system becomes
\begin{eqnarray}
\dot{z}_0 &=& p_0 v_0 z_0 - (1-p_0)v_0 z_0 - a z_0, \nonumber \\
\dot{z}_1 &=& 2 (1-p_0) v_0 z_0 + 
p_1 v_1 z_1 - (1-p_1) v_1 z_1 - a z_1, \nonumber \\
\dot{z}_2 &=& 2 (1-p_1) v_1 z_1 - d z_2 - a z_2, \nonumber \\
\dot{z}_3 &=& p_m v_m z_3 - (1-p_m) v_m z_3 - a z_3, \nonumber \\
\dot{z}_4 &=& 2 (1-p_m) v_m z_3 - d z_4 - a z_3.
\label{ode}
\end{eqnarray}
The apoptosis rate of terminal cells $d$ is a constant ($d=0.2$)
and the environmental induced death rate $a$ 
in Eq.~\eqref{eq_d} is a function of $z_i$,
$$
a = \max \left(0, \xi \left( \sum_{i=0}^{4}z_i - 1 \right) \right),
$$
where $n=z_0 + z_1 + z_2 + z_3 + z_4$ and $\xi>0$ is a constant.
$p_1 = 0.4$ and $v_1 = 1$ are constants. 
The rest of the variables in Eq.~\eqref{ode}, 
$p_0, v_0, p_m, v_m$, are also functions of $z_i$
determined by feedback rules.
The feedback rules that we have considered 
are listed in Table~\ref{feedback_schemes}.

\begin{table}
  \centering
  \caption{Feedback schemes}
  \label{feedback_schemes}
  \begin{tabular}{|p{4cm}|p{7.5cm}|p{2cm}|}
    \hline
    Feedback scheme & Feedback functions & Results appear in \\
    \hline
  TDC feedback on $p_0$.
  & $\displaystyle p_0 = \frac{1}{1.01 + k_1 z_2} $, 
  $v_0 = 1$ or 0.65, $p_m=1$, $v_m=1$. 
  & Fig.~\ref{fig_pop_struct} and Fig.~\ref{fig_fixpmvm} (a)-(f).\\
    \hline
  TDC feedback on $p_0$ and $v_0$.
  & $\displaystyle p_0 = \frac{1}{1.01 + k_1 z_2} $, 
  $\displaystyle v_0 = \left( \frac{1}{v_0^{max}} + z_2
  \left( \frac{1}{v_0^{min}} - \frac{1}{v_0^{max}}\right) \right)^{-1}$,
  $p_m=1$, $v_m=1$. 
  & Fig.~\ref{fig_self_recover} and Fig.~\ref{fig_fixpmvm} (g)-(l).\\    
     \hline
  TDC and MTDC both feedback on $p_0$ and $v_0$.
  & $\displaystyle p_0 = \frac{1}{1.01 + k_1 (z_2+z_4)} $, 
  $\displaystyle v_0 = \left( \frac{1}{v_0^{max}} + (z_2+z_4)
  \left( \frac{1}{v_0^{min}} - \frac{1}{v_0^{max}}\right) \right)^{-1}$,
  $p_m=1$, $v_m=1$. 
  & Fig.~\ref{fig_fixpmvm} (m)-(r).\\    
     \hline
  TDC feedback on $p_0, v_0$ and $p_m, v_m$.
  & $\displaystyle p_0 = \frac{1}{1.01 + k_1 z_2} $, 
  $\displaystyle v_0 = \left( \frac{1}{v_0^{max}} + z_2
  \left( \frac{1}{v_0^{min}} - \frac{1}{v_0^{max}}\right) \right)^{-1}$,
  $\displaystyle p_m = \frac{1}{1.01 + k_m z_2} $, 
  $\displaystyle v_m = \left( \frac{1}{v_m^{max}} + z_2
  \left( \frac{1}{v_m^{min}} - \frac{1}{v_m^{max}}\right) \right)^{-1}$.
  & Fig.~\ref{fig_feedpmvm} (a)-(b).\\
     \hline
  TDC and MTDC both feedback on $p_0, v_0$ and $p_m, v_m$.
  & $\displaystyle p_0 = \frac{1}{1.01 + k_1 (z_2+z_4)} $, 
  $\displaystyle v_0 = \left( \frac{1}{v_0^{max}} + (z_2+z_4)
  \left( \frac{1}{v_0^{min}} - \frac{1}{v_0^{max}}\right) \right)^{-1}$,
  $\displaystyle p_m = \frac{1}{1.01 + k_m (z_2+z_4)} $, 
  $\displaystyle v_m = \left( \frac{1}{v_m^{max}} + (z_2+z_4)
  \left( \frac{1}{v_m^{min}} - \frac{1}{v_m^{max}}\right) \right)^{-1}$.
  & Fig.~\ref{fig_feedpmvm} (c)-(d).\\
     \hline
  \end{tabular}
\end{table}

As a special case when $d=0$ means there is no environmental
induced death, and without mutation cell, 
the ODE system Eq.~\eqref{ode} becomes
\begin{eqnarray*}
\dot{z}_0 &=& p_0 v_0 z_0 - (1-p_0)v_0z_0, \nonumber \\
\dot{z}_1 &=& 2 (1-p_0) v_0 z_0 + 
p_1 v_1 z_1 - (1-p_1) v_1 z_1, \nonumber \\
\dot{z}_2 &=& 2 (1-p_1) v_1 z_1 - d z_2.
\end{eqnarray*}
This is the same with the model studied in~\cite{Arthur09}.

\subsection*{Parameters Study}

\subsubsection*{$N$, the carrying capacity}
The carrying capacity provides a ceiling to the total population
size. When $N$ is small, the stochastic fluctuation is more
obvious. For example, the small SC population is more likely 
to jump to zero and thus leads to the extinguish of the cell 
lineage. When $N$ is large, the stochastic fluctuation become
relatively small compared with the population size. The SC 
population is less likely to self-extinguish. When $N$ become
very large, the behavior of the stochastic model approaches to
the deterministic model, which is independent of $N$.

In Fig.~\ref{afig_paraN} we let $N$ change while keep the other
parameters the same as Table~\ref{table_paras}. 
The initial condition for the stochastic model is 
$\mathbf{X}_0 = (10, 0, 0, 0, 0)$ for all cases and the initial 
condition for the deterministic case in Eq.~\eqref{ode} is 
$\mathbf{Z}_0 = (0.05, 0, 0, 0, 0)$. At time $t=150$ we
introduced one mutant cell into the system (or set $z_3=0.005$ 
in the deterministic case). The simulated trajectories
during time $[0, 300]$ are shown in the figure. 
For $N = 20$ the trajectory experiences a lot of noise and 
the cell lineage went self-extinguish later. As $N$ grows
the noise become relative small and for $N=2\times 10^4$ the
dynamic of the stochastic system become very similar to that of
the deterministic system. In this case, the mutant cells have not
been completely removed from the system and they may
revival again. In the deterministic case, the value of $z_3$
can become extremely small but still not zero, so they can
also revival.
 
\begin{figure}
\centering
\subfigure[]{
\includegraphics[width=0.45\textwidth]{calfigure/afig_paraN1.eps}
}
\subfigure[]{
\includegraphics[width=0.45\textwidth]{calfigure/afig_paraN2.eps}
}
\subfigure[]{
\includegraphics[width=0.45\textwidth]{calfigure/afig_paraN3.eps}
}
\subfigure[]{
\includegraphics[width=0.45\textwidth]{calfigure/afig_paraN4.eps}
}
\subfigure[]{
\includegraphics[width=0.45\textwidth]{calfigure/afig_paraode.eps}
}
\caption{Parameter $N$}
\label{afig_paraN}
\end{figure}

\subsection*{$\xi$, the rigidness of ceiling}
In our model, when the total population grow over the
carrying capacity $N$, all the cells will experience an
environmental induced death rate given by Eq.~\eqref{eq_d}.
Think of the competing for living environment between cells
as some kind of pressure, then the parameter $\xi$ in 
Eq.~\eqref{eq_d} describes the elasticity of the 
population ceiling.

For $N=200$ in the stochastic case 
and the deterministic case, simulated
trajectories for $\xi = 20, 200, 2000$ is shown in
Fig.~\ref{afig_paraxi}. The parameter setting
and the initial value are the same as above.
In the stochastic case, 
as $\xi$ goes large, the ceiling becomes
more flat. In the limit of $\xi\rightarrow\infty$,
our model approaches to the ``solid ceiling'' model~\cite{Lindenmayer95}
in which one randomly chosen cell is removed as soon as
the total population exceeds the carrying capacity.
We notice that for large $\xi$ the SC population tend 
to stay up for longer period of time. Both the 
ceiling behavior and the SC activation time
is consistent in the deterministic model. In fact,
as $\xi$ goes larger and larger the ODE system become
more and more stiff. 

\begin{figure}
\centering
\subfigure[]{
\includegraphics[width=0.45\textwidth]{calfigure/afig_paraxi1.eps}
}
\subfigure[]{
\includegraphics[width=0.45\textwidth]{calfigure/afig_paraodexi1.eps}
}
\subfigure[]{
\includegraphics[width=0.45\textwidth]{calfigure/afig_paraxi2.eps}
}
\subfigure[]{
\includegraphics[width=0.45\textwidth]{calfigure/afig_paraodexi2.eps}
}
\subfigure[]{
\includegraphics[width=0.45\textwidth]{calfigure/afig_paraxi3.eps}
}
\subfigure[]{
\includegraphics[width=0.45\textwidth]{calfigure/afig_paraodexi3.eps}
}
\caption{Parameter $\xi$}
\label{afig_paraxi}
\end{figure}

\subsection*{$p_1$, the self-renewal probability of TAC}
We conduct linear stability analysis for the ODE model 
without the mutant type (the first three equations in Eq.~\eqref{ode}).
While fixing the other parameters as Table~\ref{table_paras}, 
the population of SC, TAC, and TDC at the steady state for 
different $p_1$ is shown in Fig.~\ref{afig_stab_p1} (a). 
It shows as $p_1$ increases, the proportion of TAC and TDC 
grows while the proportion of SC decrease. 
We also conduct linear stability analysis around the
steady state and it shows that the eigenvalues of the Jacobian
matrix at the steady state are all real and negative, 
as shown in (b). One can also see
that the system reaches the steady state during time $[0, 150]$ 
before the mutation cell is added in the ODE trajectories in (e)-(h).

At time $t=150$, a mutant cell with $p_m=v_m=1$ is added to the system. 
This is done by change $x_3$ from 0 to a small positive number.
Now in the new ODE system as given by Eq.~\eqref{ode}, 
the previous stationary state is not stable anymore because 
of the mutant cell. The new steady state is shown in (c) 
which is different from (a).
Fig.~\ref{afig_stab_p1} (d) shows the real and imaginary part
of the eigenvalue of the Jacobian matrix computed at this 
steady state. When $p_1$ is small,
the real part is less than 0, and
the steady state is locally stable,   
the ODE system will oscillate for a while around the steady
state and reach the steady state, as shown in (e) and (f).
When $p_1$ is large,
the real part is larger than 0, and
the steady state is locally non-stable,   
the ODE system will oscillate forever, as shown in (g) and (h).

This is consistent with the stochastic
model ( (i)-(l) ), where increase of $p_1$ also leads to increase of the TAC
population and making the time of self-recovery longer. 
The difference is that in the stochastic case, the 
system will not come to a steady state as $t\rightarrow \infty$.
Also note that, when $p_1=0$, the normal lineage and 
the mutation cell coexistent, while for larger $p_1=0.2$ or larger
the self-recovery can happen and drive the mutant cell to 
extinguish. However, due to the randomness of the process, 
there is no such a critical $p_1$ that markers the 
transition between coexistence and self-recovery.
Finally, we note that when $p_1$ is larger than $0.54$,
the SC population will die out and the TAC population
can self-sustain, give rise to a two-species lineage system.

\begin{figure}
\centering
\subfigure[$p_1$ and the steady state for three species system]{
\includegraphics[width=0.45\textwidth]{calfigure/afig_stab_p1.eps}
}
\subfigure[Eigenvalue of the Jacobian matrix at the steady state]{
\includegraphics[width=0.45\textwidth]{calfigure/afig_eigenvalue_vs_p1_3p.eps}
}
\subfigure[$p_1$ and the steady state for five species system]{
\includegraphics[width=0.45\textwidth]{calfigure/afig_stab_p1_5p.eps}
}
\subfigure[Eigenvalue of the Jacobian matrix at the steady state]{
\includegraphics[width=0.45\textwidth]{calfigure/afig_eigenvalue_vs_p1.eps}
}
\subfigure[ODE trajectory at A]{
\includegraphics[width=0.2\textwidth]{calfigure/afig_stab_p1A.eps}
}
\subfigure[ODE trajectory at B]{
\includegraphics[width=0.2\textwidth]{calfigure/afig_stab_p1B.eps}
}
\subfigure[ODE trajectory at C]{
\includegraphics[width=0.2\textwidth]{calfigure/afig_stab_p1C.eps}
}
\subfigure[ODE trajectory at D]{
\includegraphics[width=0.2\textwidth]{calfigure/afig_stab_p1D.eps}
}
\subfigure[Stochastic trajectory at A]{
\includegraphics[width=0.2\textwidth]{calfigure/afig_para_p1_00.eps}
}
\subfigure[Stochastic trajectory at B]{
\includegraphics[width=0.2\textwidth]{calfigure/afig_para_p1_02.eps}
}
\subfigure[Stochastic trajectory at C]{
\includegraphics[width=0.2\textwidth]{calfigure/afig_para_p1_04.eps}
}
\subfigure[Stochastic trajectory at D]{
\includegraphics[width=0.2\textwidth]{calfigure/afig_para_p1_05.eps}
}
\caption{Parameter $p_1$}
\label{afig_stab_p1}
\end{figure}


\subsection*{$p_0$, the self-renewal probability of SC}
The self-renewal probability of SC determined by Eq.~\eqref{feedp0}
is crucial to the dynamics of the system. Here we study the
functional form of $p_0$ given by
$$
p_0 = \frac{1}{1.01+ (k_0 x_2 / N)^\eta},
$$
and examine how parameters $k_0$ and $\eta$ effect the system
kinetics.

In the following simulation, we choose relative large $N=500$ and
asymmetric division so that the system is less likely to 
self-extinguish. While keeping other parameters fixed as in 
Table~\ref{table_paras}, we search the parameter space of $k_0$
and $\eta$ that can have robust self-recovery process. For
each fixed $k_0$ and $\eta$, we simulate the system to $T=20000$.
During one simulation, we keep add mutant cells after every 500
units of time. In all 39 mutant cells is added and we count how
many times the system can recover from the mutation.
For each $k_0$ and $\eta$, the function plot of $p_0$
is shown in Fig.~\ref{afig_p0} (a). The blue region is
all the function of $p_0$ we have used. The green curve
is the feedback function Eq.~\eqref{feedp0} we used in the
main article. The purple region is the feedback function 
under which the system can self-recovery more than 37 times out from 39.

Next we study how $k_0$ and $\eta$ are related with the time 
length of self-recovery. We define the length of each recovery 
event as the time it spends from the introducing of MC to the 
next time when no MC in the system and the number of SC drops 
below 200 (the population capacity is 500). Fig.~\ref{afig_p0} (b) 
shows the average self-recovery time at each point 
in the parameter space. The white region means the 
system cannot self-recovery, either because the feedback 
is too strong so that drives the system to extinguish 
quickly or the feedback is too weak so that SCs are always active. 
Also one can see that as $k_0$ and $\eta$ get larger, 
the feedback gets stronger, and the system is more likely to extinguish. 
The region of parameters that allow self-recovery seems 
to locate in a narrow strip and the average recovery time 
tend to be smaller for larger $k_0$.

\begin{figure}
\centering
\subfigure[feedback function]{
\includegraphics[width=0.45\textwidth]{calfigure/afig_p0eta.eps}
}
\subfigure[feedback length]{
\includegraphics[width=0.45\textwidth]{calfigure/afig_p0length.eps}
}
\caption{Parameter $p_0$}
\label{afig_p0}
\end{figure}

\subsection*{$v_0$, the division of SC}
Now we change the parameters $v_0^{max}$ and $v_0^{min}$ 
that control the feedback function on $v_0$ in Eq.~\eqref{feedv0}. 
In Fig.~\ref{afig_v0}, it seems if $v_0^{max}$ or $v_0^{min}$ increases, 
the parameter region that allows self-recovery will 
shift to the left a little bit. An explanation for this 
is that with a larger $v_0$, the self-renewal probability $p_0$ 
will be smaller accordingly.


\begin{figure}
\centering
\subfigure[]{
\includegraphics[width=0.45\textwidth]{calfigure/afig_v0_1.eps}
}
\subfigure[]{
\includegraphics[width=0.45\textwidth]{calfigure/afig_v0_2.eps}
}
\subfigure[]{
\includegraphics[width=0.45\textwidth]{calfigure/afig_v0_3.eps}
}
\subfigure[]{
\includegraphics[width=0.45\textwidth]{calfigure/afig_v0_4.eps}
}
\caption{Parameter $v_0$}
\label{afig_v0}
\end{figure}


\section*{Asymmetric division vs symmetric division}
As mentioned in the main article, even thought the $p_0$
and $p_1$ are fixed, we still have some degree of freedom 
in choosing $q_0, r_0$ and $q_1, r_1$. Effectively, this is
change the preference between asymmetrical division and 
symmetrical division. In Fig.~\ref{afig_asm} we have four
sets of typical trajectories corresponding to cases where 
(a) asymmetrical division is preferred by 
both divisions in SC and TAC; (b) asymmetrical division 
is preferred by SC but not TAC; (c) asymmetrical division 
is preferred by TAC but not SC; (d) symmetrical division
is preferred by both divisions in SC and TAC. The parameters
are choosing as Table~\ref{table_paras}.

We can see that (a) has the least random fluctuation while
(d) has the most, indicating asymmetrical division help
to reduce noise, thus making the system less likely to go
self-extinguish. But as $N$ goes to infinity, the difference 
between asymmetrical and symmetrical division will be gone 
since they all approach to the same ODE system.

\begin{figure}
\centering
\subfigure[SC - asym, TAC - asym]{
\includegraphics[width=0.45\textwidth]{calfigure/fig_logi_selfrecover_aa.eps}
}
\subfigure[SC - asym, TAC - sym]{
\includegraphics[width=0.45\textwidth]{calfigure/fig_logi_selfrecover_as.eps}
}
\subfigure[SC - sym, TAC - asym]{
\includegraphics[width=0.45\textwidth]{calfigure/fig_logi_selfrecover_sa.eps}
}
\subfigure[SC - sym, TAC - sym]{
\includegraphics[width=0.45\textwidth]{calfigure/fig_logi_selfrecover_ss.eps}
}
\caption{Asymmetrical vs symmetrical division}
\label{afig_asm}
\end{figure}


\section*{Defective terminal mutant cells}
In Fig.~\ref{fig_feedpmvm} when we assume the differentiated 
mutant cell can function as a normal TDC to feedback on SC, 
we treat a differentiated mutant cell as exactly the same
as a normal TDC in its ability to feedback. But it might be
the case that there is some defective in the differentiated 
mutant cells such that they are not as good as normal TDC.
Now the feedback source should be modified as
$$
\text{Feedback source} = \text{TDC} + 
e\times \text{differentiated mutant cell}
= x_2 + e x_4,
$$
where $e<1$ is the effectiveness of the differentiated mutant
cell in feedback.
Fig.~\ref{afig_defect} shows the takeover of the mutation with 
defective differentiated mutant cell. 
We choose two cases $e=0.3$ and $e=0.5$ for both one 
feedback on $p_0$ case and feedback on both $p_0$ and $v_0$
cases. All other settings are the same as Fig.~\ref{fig_feedpmvm}.
It can be seen that the defective in differentiated mutant cell
greatly reduce the ability to takeover in the two feedback case.

\begin{figure}
\centering
\subfigure[1 feedback, $e = 0.3$, takeover]{
\includegraphics[width=0.4\textwidth]{calfigure/pmvm_takeover_1f03a.eps}
}
\subfigure[1 feedback, $e = 0.3$, response]{
\includegraphics[width=0.4\textwidth]{calfigure/pmvm_response_1f03a.eps}
}
\subfigure[1 feedback, $e = 0.5$, takeover]{
\includegraphics[width=0.4\textwidth]{calfigure/pmvm_takeover_1f05a.eps}
}
\subfigure[1 feedback, $e = 0.5$, response]{
\includegraphics[width=0.4\textwidth]{calfigure/pmvm_response_1f05a.eps}
}
\subfigure[2 feedback, $e = 0.3$, takeover]{
\includegraphics[width=0.4\textwidth]{calfigure/pmvm_takeover_2f03a.eps}
}
\subfigure[2 feedback, $e = 0.3$, response]{
\includegraphics[width=0.4\textwidth]{calfigure/pmvm_response_2f03a.eps}
}
\subfigure[2 feedback, $e = 0.5$, takeover]{
\includegraphics[width=0.4\textwidth]{calfigure/pmvm_takeover_2f05a.eps}
}
\subfigure[2 feedback, $e = 0.5$, response]{
\includegraphics[width=0.4\textwidth]{calfigure/pmvm_response_2f05a.eps}
}
\caption{Parameter $v_0$}
\label{afig_defect}
\end{figure}


\begin{figure}
\centering
\includegraphics[width=\textwidth]{artfigure/figure1.eps}
\caption{population structure under feedback only on $p_0$.}
\label{fig_feedback}
\end{figure}

\begin{figure*}
\centering
\subfigure[$k_0=0.6$]{
\includegraphics[width=0.45\textwidth]{calfigure/fig_1faa06.eps}
}
\subfigure[$k_0=1$]{
\includegraphics[width=0.45\textwidth]{calfigure/fig_1faa10.eps}
}
\subfigure[$k_0=0.6$]{
\includegraphics[width=0.45\textwidth]{calfigure/fig_1faa06_pvplot.eps}
}
\subfigure[$k_0=1$]{
\includegraphics[width=0.45\textwidth]{calfigure/fig_1faa10_pvplot.eps}
}
\subfigure[Average]{
\includegraphics[width=0.45\textwidth]{calfigure/fig_k1avarage_1f.eps}
}
\caption{population structure under feedback on $p_0$.}
\label{fig_pop_struct}
\end{figure*}

\begin{figure*}
\centering
\subfigure[$k_0=0.6, v_0^{min} = 0.5, v_0^{max}=3.0$]{
\includegraphics[width=0.45\textwidth]{calfigure/fig_2faa06.eps}
}
\subfigure[$k_0=1, v_0^{min} = 0.5, v_0^{max}=3.0$]{
\includegraphics[width=0.45\textwidth]{calfigure/fig_2faa10.eps}
}
\subfigure[$k_0=0.6, v_0^{min} = 0.5, v_0^{max}=3.0$]{
\includegraphics[width=0.45\textwidth]{calfigure/fig_2faa06_pvplot.eps}
}
\subfigure[$k_0=1, v_0^{min} = 0.5, v_0^{max}=3.0$]{
\includegraphics[width=0.45\textwidth]{calfigure/fig_2faa10_pvplot.eps}
}
\subfigure[Average]{
\includegraphics[width=0.45\textwidth]{calfigure/fig_k1avarage_2f.eps}
}
\caption{population structure with feedback on $p_0$ and $v_0$.}
\label{fig_pop_struct_two}
\end{figure*}

\begin{figure*}
\centering
\subfigure[Self-recovery]{
\includegraphics[width=0.45\textwidth]{calfigure/fig_selfrecover.eps}
}
\subfigure[zoom in]{
\includegraphics[width=0.45\textwidth]{calfigure/fig_selfrecover_zoomin.eps}
}
\subfigure[$p_0$, $v_0$ values]{
\includegraphics[width=0.5\textwidth]{calfigure/fig_selfrecover_pvplot.eps}
}
\caption{Self-recovery}
\label{fig_self_recover}
\end{figure*}

\begin{figure}
\centering
\includegraphics[width=\textwidth]{artfigure/figure4.eps}
\caption{Self-recovery}
\label{fig_cartoon}
\end{figure}

\begin{figure*}
\centering
\subfigure[one feedback, takeover]{
\includegraphics[width=0.3\textwidth]{calfigure/pmvm_takeover_1fna.eps}
}
\subfigure[SC response]{
\includegraphics[width=0.3\textwidth]{calfigure/pmvm_response_1fna.eps}
}
\subfigure[response curve]{
\includegraphics[width=0.3\textwidth]{calfigure/pmvm_scnumber_1fna.eps}
}
\subfigure[typical trajectory A]{
\includegraphics[width=0.3\textwidth]{calfigure/pmvm_point_1fA.eps}
}
\subfigure[typical trajectory B]{
\includegraphics[width=0.3\textwidth]{calfigure/pmvm_point_1fB.eps}
}
\subfigure[typical trajectory C]{
\includegraphics[width=0.3\textwidth]{calfigure/pmvm_point_1fC.eps}
}
\subfigure[two feedbacks, takeover]{
\includegraphics[width=0.3\textwidth]{calfigure/pmvm_takeover_2fna.eps}
}
\subfigure[SC response]{
\includegraphics[width=0.3\textwidth]{calfigure/pmvm_response_2fna.eps}
}
\subfigure[response curve]{
\includegraphics[width=0.3\textwidth]{calfigure/pmvm_scnumber_2fna.eps}
}
\subfigure[typical trajectory A]{
\includegraphics[width=0.3\textwidth]{calfigure/pmvm_point_2fA.eps}
}
\subfigure[typical trajectory B]{
\includegraphics[width=0.3\textwidth]{calfigure/pmvm_point_2fB.eps}
}
\subfigure[typical trajectory C]{
\includegraphics[width=0.3\textwidth]{calfigure/pmvm_point_2fC.eps}
}
\subfigure[two feedbacks, takeover]{
\includegraphics[width=0.3\textwidth]{calfigure/pmvm_takeover_2fma.eps}
}
\subfigure[SC response]{
\includegraphics[width=0.3\textwidth]{calfigure/pmvm_response_2fma.eps}
}
\subfigure[response curve]{
\includegraphics[width=0.3\textwidth]{calfigure/pmvm_scnumber_2fma.eps}
}
\subfigure[typical trajectory A]{
\includegraphics[width=0.3\textwidth]{calfigure/pmvm_point_2fmA.eps}
}
\subfigure[typical trajectory B]{
\includegraphics[width=0.3\textwidth]{calfigure/pmvm_point_2fmB.eps}
}
\subfigure[typical trajectory C]{
\includegraphics[width=0.3\textwidth]{calfigure/pmvm_point_2fmC.eps}
}
\caption{Self-recovery}
\label{fig_fixpmvm}
\end{figure*}


\begin{figure*}
\centering
\subfigure[take over, mutant cell does not feedback]{
\includegraphics[width=0.45\textwidth]{calfigure/f2m_n.eps}
}
\subfigure[SC response, mutant cell does not feedback]{
\includegraphics[width=0.45\textwidth]{calfigure/f2m_n_scresponse.eps}
}
\subfigure[takeover, mutant cell feedback]{
\includegraphics[width=0.45\textwidth]{calfigure/f2m_m.eps}
}
\subfigure[SC response, mutant cell feedback]{
\includegraphics[width=0.45\textwidth]{calfigure/f2m_m_scresponse.eps}
}
\caption{Self-recovery}
\label{fig_feedpmvm}
\end{figure*}

\section*{Mutation Study}

\subsection*{Experiment 1}
For $N=200$ and $N=2000$, simulate 1000 samples of trajectories
of system evolution with one mutation cell added to the system
at time $t=150$. When $t=400$ the simulation ends and the system
state is recorded. We call it a mutation-free event if by then 
there is no mutation in the system. The sampled probability
of non-mutation-free events for $N=200$ is 0.467 while 
for $N=2000$ is 0.049. In this case $N$ increases by one fold, 
and the system is more rubust to mutations.

\subsection*{Experiment 2}
For $N=200$ simulate two sets of 1000 samples of trajectories.
In set 2, when adding the mutation cell at time $t=150$,
we also cut the population of each species by 1/4. 
When $t=400$ the simulation ends and the system
state is recorded. We recorded 13 times of
takeover for set 1 and only 1 time of takeover 
for set 2. This suggest a wound response at the
mutation will not help mutation cell to takeover
in the feedback system.

\end{document}
