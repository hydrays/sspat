\documentclass[12pt]{article}
\usepackage{graphicx}
\usepackage{amscd}
\usepackage{amssymb}
\usepackage{mathrsfs}
\usepackage{amsfonts}
\usepackage{mathtools}
%\usepackage{subfigure}

\newcommand{\bfmath}[1]{{\mathchoice
    {\mbox{\boldmath$\displaystyle#1$}}
    {\mbox{\boldmath$\textstyle#1$}}
    {\mbox{\boldmath$\scriptstyle#1$}}
    {\mbox{\boldmath$\scriptscriptstyle#1$}}}}
\usepackage[T1]{fontenc}
\usepackage[latin1]{inputenc}
\usepackage[a4paper]{geometry}
%\usepackage{lmodern}
\usepackage[english]{babel}
\newcommand{\xb}{\bfmath{x}}
\newcommand{\Xb}{\bfmath{X}}

\newtheorem{model}{Model}

%double line
\renewcommand{\baselinestretch}{1.5}

\begin{document}
\title{Feedback Regulation in 1D-PDE}
\maketitle

\section{Model}

Reaction-diffusion equation for the concentration 
of SC ($\phi_0(x)$), TC ($\phi_1(x)$) and MC ($\phi_m(x)$).
\begin{eqnarray*}
\frac{\partial \phi_0}{\partial t} &=&
D_0 \frac{\partial^2 \phi_0}{\partial x^2} + S_0, \\
\frac{\partial \phi_1}{\partial t} &=&
D_1 \frac{\partial^2 \phi_1}{\partial x^2} + S_1, \\
\frac{\partial \phi_m}{\partial t} &=&
D_m \frac{\partial^2 \phi_m}{\partial x^2} + S_m, \\
\frac{\partial \phi}{\partial x}|_{0, L} &=& 0.
\end{eqnarray*}
The source terms are
\begin{eqnarray*}
S_0 &=& q v_0 (2p_0 - 1) \phi_0 - d \phi_0, \\
S_1 &=& q v_0 2(1 - p_0) \phi_0 - a \phi_1, \\
S_m &=& q v_m (2p_m - 1) \phi_m - d \phi_m.
\end{eqnarray*}
The feedback from TC to SC is via TGF ($c(x)$),
which is in quasi-steady state given by
$$
0 = D_c \frac{\partial^2 c}{\partial x^2} + 
\alpha \phi_1 - \beta c, \frac{\partial c}{\partial x}|_{x=0} = 0, 
\frac{\partial c}{\partial x}|_{x=L} = 0.2.
$$
where $\alpha$ is the production rate of TGF by TC and
$\beta$ is the auto-decay rate of TGF.
Given $c(x)$, $p_0$ and $v_0$ is determined by
\begin{eqnarray*}
p_0 &=& 1 / (1.01 + k_1 c), \\
v_0 &=& v_0^{max} / (1 + k_2 c).
\end{eqnarray*}
Now we define the pressure as
$$
\eta = \phi_0 + \phi_m.
$$
This pressure effect the overall death rate $d$ 
and the division rate $q$ by
\begin{eqnarray*}
d &=& \max \{0, \xi (\eta - \eta_0)\}, \\
q &=& 1 / (1 + e^{\gamma (\eta - \eta_0)} ).
\end{eqnarray*}

\section{Numerical experiments}
\noindent\textbf{Design principle}\\
The source term of SC and MC are controlled by $q$ and $d$ by
$$
S = q v_0 (2p_0 - 1) \phi - d \phi.
$$
Since MC is always advantageous than SC at the same location,
we choose $q$ to be almost zero around MC region.
Thus the MC get killed while the SC can be supplied by the 
diffusion of remote SCs. See Fig.~1.

\begin{figure}
\includegraphics[width=\textwidth]{fig1.eps}
\end{figure}

\noindent\textbf{Current status}\\
See movie 1.

At time 100 we introduce a small amount of MC on the left-hand-side
boundary and let it grow for a while. At time 120 we give
an extra decay term on TC, so the source term for $\phi_1$ now is
$$
S_1 = qv_02(1-p_0)\phi_0 - d\phi_0 - 100\phi_m.
$$

Because the MC is more advantageous than SC in region B, in 
the continous model a tiny amount of MC can spread via diffusion
to B and proliferate there, causing the ``tunnelling effect''.
So we modify $\phi_m$ by setting $\phi_m = 0$ when its value
is below a certain threshold, say $tol = 10^{-7}$.

Now we have movie 2.

\noindent\textbf{On-going study}\\
From here, we have been trying to remove the extra source term
in $S_1$, i.e., not assuming MC consumes TC. It seems the shape
of TC (or TGF) around the mutation site is important 
(as it controls $v_0$ and $p_0$).

\begin{itemize}
\item
The first experiment is to reduce the coefficient
in $100\phi_m$. If it is too small, the MC will begin
to spread. 

\item
We tried to make $p_0$ and $v_0$ to be more sensitive 
to TGF. For example, using higher exponential terms in
the Hill's function. But doing that may leads to instability issues.
See movie 3.
\end{itemize} 


\end{document}
